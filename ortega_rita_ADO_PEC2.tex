% Options for packages loaded elsewhere
\PassOptionsToPackage{unicode}{hyperref}
\PassOptionsToPackage{hyphens}{url}
%
\documentclass[
  english,
]{article}
\usepackage{lmodern}
\usepackage{amssymb,amsmath}
\usepackage{ifxetex,ifluatex}
\ifnum 0\ifxetex 1\fi\ifluatex 1\fi=0 % if pdftex
  \usepackage[T1]{fontenc}
  \usepackage[utf8]{inputenc}
  \usepackage{textcomp} % provide euro and other symbols
\else % if luatex or xetex
  \usepackage{unicode-math}
  \defaultfontfeatures{Scale=MatchLowercase}
  \defaultfontfeatures[\rmfamily]{Ligatures=TeX,Scale=1}
\fi
% Use upquote if available, for straight quotes in verbatim environments
\IfFileExists{upquote.sty}{\usepackage{upquote}}{}
\IfFileExists{microtype.sty}{% use microtype if available
  \usepackage[]{microtype}
  \UseMicrotypeSet[protrusion]{basicmath} % disable protrusion for tt fonts
}{}
\makeatletter
\@ifundefined{KOMAClassName}{% if non-KOMA class
  \IfFileExists{parskip.sty}{%
    \usepackage{parskip}
  }{% else
    \setlength{\parindent}{0pt}
    \setlength{\parskip}{6pt plus 2pt minus 1pt}}
}{% if KOMA class
  \KOMAoptions{parskip=half}}
\makeatother
\usepackage{xcolor}
\IfFileExists{xurl.sty}{\usepackage{xurl}}{} % add URL line breaks if available
\IfFileExists{bookmark.sty}{\usepackage{bookmark}}{\usepackage{hyperref}}
\hypersetup{
  pdftitle={PEC2 Análisis de Datos Ómicos},
  pdfauthor={Rita Ortega Vallbona},
  hidelinks,
  pdfcreator={LaTeX via pandoc}}
\urlstyle{same} % disable monospaced font for URLs
\usepackage[margin=1in]{geometry}
\usepackage{color}
\usepackage{fancyvrb}
\newcommand{\VerbBar}{|}
\newcommand{\VERB}{\Verb[commandchars=\\\{\}]}
\DefineVerbatimEnvironment{Highlighting}{Verbatim}{commandchars=\\\{\}}
% Add ',fontsize=\small' for more characters per line
\usepackage{framed}
\definecolor{shadecolor}{RGB}{248,248,248}
\newenvironment{Shaded}{\begin{snugshade}}{\end{snugshade}}
\newcommand{\AlertTok}[1]{\textcolor[rgb]{0.94,0.16,0.16}{#1}}
\newcommand{\AnnotationTok}[1]{\textcolor[rgb]{0.56,0.35,0.01}{\textbf{\textit{#1}}}}
\newcommand{\AttributeTok}[1]{\textcolor[rgb]{0.77,0.63,0.00}{#1}}
\newcommand{\BaseNTok}[1]{\textcolor[rgb]{0.00,0.00,0.81}{#1}}
\newcommand{\BuiltInTok}[1]{#1}
\newcommand{\CharTok}[1]{\textcolor[rgb]{0.31,0.60,0.02}{#1}}
\newcommand{\CommentTok}[1]{\textcolor[rgb]{0.56,0.35,0.01}{\textit{#1}}}
\newcommand{\CommentVarTok}[1]{\textcolor[rgb]{0.56,0.35,0.01}{\textbf{\textit{#1}}}}
\newcommand{\ConstantTok}[1]{\textcolor[rgb]{0.00,0.00,0.00}{#1}}
\newcommand{\ControlFlowTok}[1]{\textcolor[rgb]{0.13,0.29,0.53}{\textbf{#1}}}
\newcommand{\DataTypeTok}[1]{\textcolor[rgb]{0.13,0.29,0.53}{#1}}
\newcommand{\DecValTok}[1]{\textcolor[rgb]{0.00,0.00,0.81}{#1}}
\newcommand{\DocumentationTok}[1]{\textcolor[rgb]{0.56,0.35,0.01}{\textbf{\textit{#1}}}}
\newcommand{\ErrorTok}[1]{\textcolor[rgb]{0.64,0.00,0.00}{\textbf{#1}}}
\newcommand{\ExtensionTok}[1]{#1}
\newcommand{\FloatTok}[1]{\textcolor[rgb]{0.00,0.00,0.81}{#1}}
\newcommand{\FunctionTok}[1]{\textcolor[rgb]{0.00,0.00,0.00}{#1}}
\newcommand{\ImportTok}[1]{#1}
\newcommand{\InformationTok}[1]{\textcolor[rgb]{0.56,0.35,0.01}{\textbf{\textit{#1}}}}
\newcommand{\KeywordTok}[1]{\textcolor[rgb]{0.13,0.29,0.53}{\textbf{#1}}}
\newcommand{\NormalTok}[1]{#1}
\newcommand{\OperatorTok}[1]{\textcolor[rgb]{0.81,0.36,0.00}{\textbf{#1}}}
\newcommand{\OtherTok}[1]{\textcolor[rgb]{0.56,0.35,0.01}{#1}}
\newcommand{\PreprocessorTok}[1]{\textcolor[rgb]{0.56,0.35,0.01}{\textit{#1}}}
\newcommand{\RegionMarkerTok}[1]{#1}
\newcommand{\SpecialCharTok}[1]{\textcolor[rgb]{0.00,0.00,0.00}{#1}}
\newcommand{\SpecialStringTok}[1]{\textcolor[rgb]{0.31,0.60,0.02}{#1}}
\newcommand{\StringTok}[1]{\textcolor[rgb]{0.31,0.60,0.02}{#1}}
\newcommand{\VariableTok}[1]{\textcolor[rgb]{0.00,0.00,0.00}{#1}}
\newcommand{\VerbatimStringTok}[1]{\textcolor[rgb]{0.31,0.60,0.02}{#1}}
\newcommand{\WarningTok}[1]{\textcolor[rgb]{0.56,0.35,0.01}{\textbf{\textit{#1}}}}
\usepackage{graphicx,grffile}
\makeatletter
\def\maxwidth{\ifdim\Gin@nat@width>\linewidth\linewidth\else\Gin@nat@width\fi}
\def\maxheight{\ifdim\Gin@nat@height>\textheight\textheight\else\Gin@nat@height\fi}
\makeatother
% Scale images if necessary, so that they will not overflow the page
% margins by default, and it is still possible to overwrite the defaults
% using explicit options in \includegraphics[width, height, ...]{}
\setkeys{Gin}{width=\maxwidth,height=\maxheight,keepaspectratio}
% Set default figure placement to htbp
\makeatletter
\def\fps@figure{htbp}
\makeatother
\setlength{\emergencystretch}{3em} % prevent overfull lines
\providecommand{\tightlist}{%
  \setlength{\itemsep}{0pt}\setlength{\parskip}{0pt}}
\setcounter{secnumdepth}{5}
\ifxetex
  % Load polyglossia as late as possible: uses bidi with RTL langages (e.g. Hebrew, Arabic)
  \usepackage{polyglossia}
  \setmainlanguage[]{english}
\else
  \usepackage[shorthands=off,main=english]{babel}
\fi

\title{PEC2 Análisis de Datos Ómicos}
\author{Rita Ortega Vallbona}
\date{1 de junio, 2020}

\begin{document}
\maketitle

{
\setcounter{tocdepth}{3}
\tableofcontents
}
\hypertarget{abstract}{%
\section{Abstract}\label{abstract}}

\hypertarget{objetivos}{%
\section{Objetivos}\label{objetivos}}

\hypertarget{materiales-y-muxe9todos}{%
\section{Materiales y métodos}\label{materiales-y-muxe9todos}}

\hypertarget{los-datos}{%
\subsection{Los Datos}\label{los-datos}}

Los datos proporcionados en el enunciado provienen del repositorio GTEx
(\emph{Genotype-Tissue Expression}), que recoge información de expresión
específica de 54 tipos de tejido sano, proveniente de 1000 individuos.
Este \href{https://www.gtexportal.org/home/}{portal} permite el acceso a
los datos de expresión, imágenes de histología, etc.

Obtenemos los datos de targets y counts de los archivos proporcionados
en el enunciado.

Con este script extraemos 10 muestras de cada grupo del archivo
targets.csv y subseteamos las columnas escogidas en el archivo
counts.csv.

\begin{Shaded}
\begin{Highlighting}[]
\OperatorTok{>}\StringTok{ }\CommentTok{# Separamos el dataframe que recoge los targets por grupos}
\ErrorTok{>}\StringTok{ }\NormalTok{NIT <-}\StringTok{ }\KeywordTok{subset}\NormalTok{(all_targets, Group }\OperatorTok{==}\StringTok{ "NIT"}\NormalTok{)}
\OperatorTok{>}\StringTok{ }\NormalTok{SFI <-}\StringTok{ }\KeywordTok{subset}\NormalTok{(all_targets, Group }\OperatorTok{==}\StringTok{ "SFI"}\NormalTok{)}
\OperatorTok{>}\StringTok{ }\NormalTok{ELI <-}\StringTok{ }\KeywordTok{subset}\NormalTok{(all_targets, Group }\OperatorTok{==}\StringTok{ "ELI"}\NormalTok{)}
\OperatorTok{>}\StringTok{ }
\ErrorTok{>}\StringTok{ }\CommentTok{# Seleccionamos 10 muestras de cada grupo y las unimos en un}
\ErrorTok{>}\StringTok{ }\CommentTok{# único #dataframe que recoge los targets con los que}
\ErrorTok{>}\StringTok{ }\CommentTok{# trabajaremos}
\ErrorTok{>}\StringTok{ }\KeywordTok{set.seed}\NormalTok{(params}\OperatorTok{$}\NormalTok{seed.extract)}
\OperatorTok{>}\StringTok{ }\NormalTok{NIT10 <-}\StringTok{ }\NormalTok{NIT[}\KeywordTok{sample}\NormalTok{(}\KeywordTok{nrow}\NormalTok{(NIT), }\DataTypeTok{size =} \DecValTok{10}\NormalTok{, }\DataTypeTok{replace =} \OtherTok{FALSE}\NormalTok{), ]}
\OperatorTok{>}\StringTok{ }\NormalTok{SFI10 <-}\StringTok{ }\NormalTok{SFI[}\KeywordTok{sample}\NormalTok{(}\KeywordTok{nrow}\NormalTok{(SFI), }\DataTypeTok{size =} \DecValTok{10}\NormalTok{, }\DataTypeTok{replace =} \OtherTok{FALSE}\NormalTok{), ]}
\OperatorTok{>}\StringTok{ }\NormalTok{ELI10 <-}\StringTok{ }\NormalTok{ELI[}\KeywordTok{sample}\NormalTok{(}\KeywordTok{nrow}\NormalTok{(ELI), }\DataTypeTok{size =} \DecValTok{10}\NormalTok{, }\DataTypeTok{replace =} \OtherTok{FALSE}\NormalTok{), ]}
\OperatorTok{>}\StringTok{ }
\ErrorTok{>}\StringTok{ }\NormalTok{mytargets <-}\StringTok{ }\KeywordTok{rbind}\NormalTok{(NIT10, SFI10, ELI10, }\DataTypeTok{deparse.level =} \DecValTok{0}\NormalTok{)}
\OperatorTok{>}\StringTok{ }
\ErrorTok{>}\StringTok{ }\CommentTok{# Extraemos los nombres de las muestras y cambiamos los}
\ErrorTok{>}\StringTok{ }\CommentTok{# guiones #por puntos para que coincidan con los nombres de}
\ErrorTok{>}\StringTok{ }\CommentTok{# las muestras en #el dataframe de counts}
\ErrorTok{>}\StringTok{ }\NormalTok{sample_names <-}\StringTok{ }\NormalTok{mytargets[, }\DecValTok{3}\NormalTok{]}
\OperatorTok{>}\StringTok{ }\NormalTok{s_names <-}\StringTok{ }\KeywordTok{gsub}\NormalTok{(}\StringTok{"-"}\NormalTok{, }\StringTok{"."}\NormalTok{, sample_names)}
\OperatorTok{>}\StringTok{ }
\ErrorTok{>}\StringTok{ }\CommentTok{# Subseteamos las columans escogidas del dataframe de counts}
\ErrorTok{>}\StringTok{ }\NormalTok{mycounts <-}\StringTok{ }\NormalTok{dplyr}\OperatorTok{::}\KeywordTok{select}\NormalTok{(all_counts, s_names)}
\end{Highlighting}
\end{Shaded}

\hypertarget{preprocesado-de-los-datos}{%
\subsection{Preprocesado de los datos}\label{preprocesado-de-los-datos}}

\hypertarget{filtraje}{%
\subsubsection{Filtraje}\label{filtraje}}

\hypertarget{normalizaciuxf3n}{%
\subsubsection{Normalización}\label{normalizaciuxf3n}}

\hypertarget{identificaciuxf3n-de-genes-diferencialmente-expresados}{%
\subsection{Identificación de genes diferencialmente
expresados}\label{identificaciuxf3n-de-genes-diferencialmente-expresados}}

\hypertarget{anotaciuxf3n-de-los-resultados}{%
\subsection{Anotación de los
resultados}\label{anotaciuxf3n-de-los-resultados}}

\hypertarget{busca-de-patrones-de-expresiuxf3n-y-agrupaciuxf3n-de-las-muestras-comparaciuxf3n-entre-las-distintas-comparaciones}{%
\subsection{Busca de patrones de expresión y agrupación de las muestras
(comparación entre las distintas
comparaciones)}\label{busca-de-patrones-de-expresiuxf3n-y-agrupaciuxf3n-de-las-muestras-comparaciuxf3n-entre-las-distintas-comparaciones}}

\hypertarget{anuxe1lisis-de-significaciuxf3n-bioluxf3gica-gene-enrichment-analysis}{%
\subsection{\texorpdfstring{Análisis de significación biológica
(``\emph{Gene Enrichment
Analysis}'')}{Análisis de significación biológica (``Gene Enrichment Analysis'')}}\label{anuxe1lisis-de-significaciuxf3n-bioluxf3gica-gene-enrichment-analysis}}

\hypertarget{resultados}{%
\section{Resultados}\label{resultados}}

\hypertarget{discusiuxf3n}{%
\section{Discusión}\label{discusiuxf3n}}

\hypertarget{apuxe9ndice}{%
\section{Apéndice}\label{apuxe9ndice}}

\hypertarget{bibliografuxeda}{%
\section{Bibliografía}\label{bibliografuxeda}}

\end{document}
